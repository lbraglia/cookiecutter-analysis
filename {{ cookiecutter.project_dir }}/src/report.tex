\documentclass{article}
\usepackage{cmbright}
%% \usepackage[a4paper, top=2cm, bottom=2cm, left=1cm, right=1cm]{geometry}
\usepackage[T1]{fontenc}
\usepackage[english]{babel}              %% inglese, progetti internazionali
%% \usepackage[english, italian]{babel}  %% roba in italiano
\usepackage[utf8]{inputenc}
\usepackage{amsmath}
\usepackage{amsfonts}
\usepackage{amssymb}
\usepackage{mypkg}
\usepackage{hyperref}
%% \extrafloats{500} % decommentare se ci sono tante tabelle
\usepackage{biblatex}
\addbibresource{proj/biblio/common_biblio.bib}
\addbibresource{proj/biblio/prj_biblio.bib}
\usepackage{pdflscape}     % per \begin{landscape} \end{landscape}
\usepackage[usefamily=R]{pythontex} % per \begin{Rcode} \end{Rcode}
\begin{document}
\title{ {{cookiecutter.pi_surname}} - {{cookiecutter.project_acronym}} }
% \date{}
% \maxdeadcycles=1000       % decommentare se ci sono tante tabelle
\maketitle
\tableofcontents
\begin{pycode}
# import numpy as np
# import pandas as pd
# import matplotlib.pyplot as plt
# import pylbmisc as lb
# from src.include import fun
# df = pd.read_pickle("tmp/clean_df.pkl")

\end{pycode}


\section{Methods}


Statistical analysis was conducted using Python \cite{python}.

\section{Results}






% ================================================================================ %
\printbibliography 
\end{document}

% 1) TABELLE: per una tabella rapida la funzione lb.io.latex_table che stampa una
% tabella (un pd.DataFrame o un oggetto inheritante avente il metodo
% to_latex). Per esportare più complessivamente (latex ed excel) usare qualcosa
% del genere
% exported = {
%     "Descrittive": desc_df,
%     "Analisi 1"  : analysis1_df,
%     "Analisi 2"  : analysis2_df
% }
% latex_refs = lb.io.export_tables(exported)

% 2) IMMAGINI: salvare l'immagine in un pycode (non visualizzato) e poi
% inserirla in latex normalmente mediante \includegraphics che potrà essere
% stampato con print (oppure incluso a mano sotto al codice python, in quello
% latex).
% lb.io.export_figure velocizza: in un environment di pycode (codice non stampato,
% per report) basta questo che salva la figura e stampa il codice latex per
% includerla
%
% fig, ax = plt.subplots()
% x = np.linspace(0, 10, 100)
% ax.plot(x, np.sin(x), label = 'sin(x)')
% ax.plot(x, np.cos(x), label = 'cos(x)')
% ax.legend()
%
% lb.io.export_figure(fig, label = 'first_plot')

% 3) FUNZIONI PYTHON IN LATEX:
% - per funzione che stampa del TeX, la definiamo in un pycode e la usiamo in pyc
%   def whatever(x):
%      ...
%   In latex \pyc{whatever(y)}
% - se in pycode definiamo una funzione whatever e in seguito in latex definiamo
%   il suo utilizzo entro py, es
%   \newcommand{\pyfunction}[2]{\py{whatever(#1, #2)}},
%   poi sempre in latex si potrà usare \pow{2}{3}


% 4) UTILIZZARE R (vedere beltrami_cogitab per qualcosa in piu)
%
% \begin{Rcode}
% source("tmp/clean_df.R")
% library(lbmisc)
% # esempio di grafici
% outfile <- "outputs/cogitab_score"
% code <- "
%     boxplot(df$score ~ df$sani_ds_mci,
%             xlab = 'Group', ylab = 'Cogitab score')
% "
% lbmisc::fig_dump(code, outfile = outfile)
% lbmisc::include_figure(outfile,
%                label = "cogitab_score",
%                caption = "Boxplot cogitab (tutti i gruppi)",
%                scale = 0.5)

% \end{Rcode}
