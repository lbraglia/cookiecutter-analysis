\documentclass{article}
\usepackage{cmbright}
%% \usepackage[a4paper, top=2cm, bottom=2cm, left=1cm, right=1cm]{geometry}
\usepackage[T1]{fontenc}
\usepackage[english]{babel}              %% inglese, progetti internazionali
%% \usepackage[english, italian]{babel}  %% roba in italiano
\usepackage[utf8]{inputenc}
\usepackage{amsmath}
\usepackage{amsfonts}
\usepackage{amssymb}
\usepackage{mypkg}
\usepackage{hyperref}
%% \extrafloats{500} % decommentare se ci sono tante tabelle
\usepackage{biblatex}
\addbibresource{proj/biblio/common_biblio.bib}
\addbibresource{proj/biblio/prj_biblio.bib}
\usepackage{pdflscape}     % per \begin{landscape} \end{landscape}
\usepackage[usefamily=R]{pythontex} % per \begin{Rcode} \end{Rcode}
\begin{document}
\title{ {{}} }
% \maxdeadcycles=1000       % decommentare se ci sono tante tabelle
\maketitle
\tableofcontents

\begin{pycode}
# import numpy as np
# import pandas as pd
# import matplotlib.pyplot as plt
# import pylbmisc as lb
# from src.include import fun
\end{pycode}




% ================================================================================ %
\printbibliography 
\end{document}
